% PREAMBLE
\documentclass[letterpaper,11pt,oneside]{article}
\usepackage[utf8]{inputenc}
\usepackage{setspace}
\usepackage{natbib}                                        % Citation package, see https://en.wikibooks.org/wiki/LaTeX/Bibliography_Management#Natbib
%\usepackage{bibentry}
%\nobibliography*
\usepackage{hyperref}
\hypersetup{colorlinks=true, linkcolor=blue, citecolor=blue, filecolor=magenta, urlcolor=blue}
\usepackage{color}
\usepackage{longtable}
\usepackage{booktabs}
\usepackage{float}
\usepackage{graphicx}
\usepackage[UKenglish]{babel}
\usepackage[UKenglish,cleanlook]{isodate}
\usepackage{geometry}
\usepackage{lastpage,fancyhdr}
%\usepackage[shortlabels]{enumitem} \setlist[enumerate]{leftmargin=0pt}
\setlength{\parindent}{0pt}

%%%%%%%%%%%%%%%%%%%%%%%%%%%%%%%%%%%%%%%%%%%%%%%%%%%%%%%%%
%% Start CV document
\begin{document}
\newgeometry{left=0.75in, right=1in, top=0.75in}
% Adjust the footer to show page counter.
\pagestyle{fancy}
\fancyhf{}
\renewcommand{\headrulewidth}{0pt}
\renewcommand{\footrulewidth}{0.4pt}
% NAME & CONTACT INFO
\centerline{\Large{\textbf{Senan Hogan-Hennessy}}}
    \begin{table}[H]
    \centering
    \begin{tabular*}{\textwidth}{l @{\extracolsep{\fill}} r}
        \toprule
        Uris Hall \#447  &
            \href{https://shoganhennessy.github.io}{
                \nolinkurl{shoganhennessy.github.io}} \\
        Economics Department &
            \href{https://economics.cornell.edu/senan-hogan-hennessy}{
                \nolinkurl{economics.cornell.edu/senan-hogan-hennessy}} \\
        Cornell University NY 14853 USA &
            \href{mailto:seh325@cornell.edu}{\nolinkurl{seh325@cornell.edu}} \\
        \bottomrule
    \end{tabular*}
\end{table}
\fancyfoot[L]{CV, Senan Hogan-Hennessy, updated \today.
    \hfill Page~$\thepage$ of \pageref{LastPage}.}

%% MAIN
% The main body is contained in a tabular environment. To move sections onto the next page, simply end the tabular environment and begin a new tabular environment.
\begin{longtable}[\textwidth]{p{0.13\textwidth} p{0.85\textwidth}}
\textbf{Doctoral Studies}
    & Cornell University \hfill (in progress) 2020--2026

    Ph.D. Economics
    
    Fields: Labour economics, applied econometrics. \\ \\
    
    & \begin{tabular}[t]{@{} l @{\hskip 1cm} l}
        \textbf{Placement Officer}     & \textbf{Placement Administrator} \\
        Levon  Barseghyan              & Jeannine Crouse Hagadorn         \\
        Robert Julius Thorne Professor & Graduate Field Coordinator       \\
        Cornell, Economics Department  & Cornell, Economics Department    \\
        \href{mailto:lb247@cornell.edu}{\nolinkurl{lb247@cornell.edu}}
            & \href{mailto:jc2298@cornell.edu}{\nolinkurl{jc2298@cornell.edu}} \\
        001 (607) 255 6284 & 001 (607) 255 4893 \\ \\
    \multicolumn{2}{l}{\hspace{-0.25cm}\textbf{Dissertation Committee and References}} \\
        Douglas Miller (Chair)                         & Zhuan Pei                      \\
        Professor, Economics                           & Associate Professor, Economics \\
        Cornell, Economics Department                  & Cornell, Economics Department  \\
        \href{mailto:dlm336@cornell.edu}{\nolinkurl{dlm336@cornell.edu}}
            & \href{mailto:zhuan.pei@cornell.edu}{\nolinkurl{zhuan.pei@cornell.edu}} \\ \\
        Evan Riehl                                     \\
        Associate Professor, Economics                 \\
        Cornell, Industrial Labour Relations \\
        \href{mailto:eriehl@cornell.edu}{\nolinkurl{eriehl@cornell.edu}}
    \end{tabular} \\ \\
\textbf{Prior}
    & Pomona College, USA \hfill 2018 \\

\textbf{Education}
    & B.A. Economics $+$ mathematics  \\ \\

\textbf{Citizenship}
    & Great Britain.%, Ireland
    \hfill \textbf{Gender:} Male. \\ \\

\textbf{Languages}
    & English (native). \\ \\
        
\textbf{Teaching Experience}
    & Intro to labour economics (undergraduate, Cornell course ILRE--2400) \hfill 2024

    \indent\hspace{1cm} Teaching assistant to Professor Stephanie Thomas.

    Applied econometrics (graduate, Cornell course ECON--6590) \hfill 2022 
    
    \indent\hspace{1cm} Teaching assistant to Professor Douglas Miller.
    
    Econometrics (undergraduate, Pomona College course ECON--167) \hfill 2017--2018
    
    \indent\hspace{1cm} Teaching assistant to Professor Pierangelo De Pace.

    Microeconomic theory (undergraduate, Pomona College course ECON--101) \hfill 2016

    \indent\hspace{1cm} Teaching assistant to Professor John Clithero. \\ \\

\textbf{Research Experience}
    & Research Assistant, Cornell University \hfill 2021--2026.

    \indent\hspace{1cm} Professor Seth Sanders (Econ), 2025--2026.

    \indent\hspace{1cm} Professor Chris Barrett (Dyson), 2023--2024.

    \indent\hspace{1cm} Professor Louis Hyman (ILR), 2021.

    Data Science Intern, The Behaviouralist (London UK) \hfill 2020

    \indent\hspace{1cm} Assistant to Professors  Robert Hahn (Oxford), Robert Metcalfe (USC).

    Research Associate, Harvard Business School \hfill 2019

    \indent\hspace{1cm} Assistant to Professor Daniel Gross. \\ \\

\textbf{Awards}
    & Research Seed Grant, Cornell Center for the Study of Inequality \hfill 2024
    
    Small Labour Grant, Cornell University ILR \hfill 2024
    
    Conference Travel Grant, Cornell University Graduate School \hfill 2023, 2024, 2025
    
    Sage Fellowship, Cornell University \hfill 2020, 2024
    
    Lorne D Cook Memorial Award, Pomona College \hfill 2018
    
    Distinction in Undergraduate Thesis, Pomona College \hfill 2018
    
    Sutton Trust --- Fulbright Programme, London UK \hfill 2013--2014 \\ \\

\textbf{Professional}
    & \textbf{Presentations:}

    European Economic Association Annual Congress, Bordeaux France \hfill 2025

    Econometric Society World Congress, Seoul Korea \hfill 2025

    European Association of Labour Economists Conference, NHH Norway \hfill 2024

    Economics Department Alumni Conference, Cornell University \hfill 2024

    Eastern Economic Association Annual Meeting, Boston USA \hfill 2024

    Center for the Study of Inequality, Cornell University \hfill 2024

    Labour Work in Progress Seminar, Cornell University \hfill 2021, 2022, and so on. \\ \\

\textbf{Research Papers}
    & \textbf{\href{https://raw.githubusercontent.com/shoganhennessy/mediation-natural-experiment/main/mediation-natural-experiment-2025.pdf}{
        Causal Mediation in Natural Experiments (2025).}}

    Natural experiments are a cornerstone of applied economics, providing settings for estimating causal effects with a compelling argument for treatment randomisation, but give little indication of the mechanisms behind causal effects.
    Causal Mediation (CM) provides a framework to analyse mechanisms by identifying the average direct and indirect effects (CM effects), yet conventional CM methods require the relevant mediator is as-good-as-randomly assigned.
    When people choose the mediator based on costs and benefits (whether to visit a doctor, to attend university, etc.), this assumption fails and conventional CM analyses are at risk of bias.
    I propose a control function strategy that uses instrumental variation in mediator take-up costs, delivering unbiased direct and indirect effects when selection is driven by unobserved gains.
    The method identifies CM effects via the marginal effect of the mediator, with parametric or semi-parametric estimation that is simple to implement in two stages.
    Applying these methods to the Oregon Health Insurance Experiment reveals a substantial portion of the Medicaid lottery's effect on self-reported health and happiness flows through increased healthcare usage --- an effect that a conventional CM analysis would mistake.
    This approach gives applied researchers an alternative method to estimate CM effects when an initial treatment is quasi-randomly assigned, but the mediator is not, as is common in natural experiments. \\ \\

    & \textbf{The Labour Market Effects of Genetics and Education (2025, JMP).}

    (Draft forthcoming)  \\ \\

    & \textbf{\href{https://raw.githubusercontent.com/shoganhennessy/state-funding-faculty/main/state-funding-faculty-2024.pdf}{
        Less Funding, More Lecturers, and Fewer Professors (2024).}}

    Public universities employ more lecturers and fewer professors today than at any other point in the last thirty years, relative to student enrolment.
    At the same time, state funding for higher education has stagnated.
    This paper shows that the decline in state funding led to a substitution away from professors toward lecturers at US public universities.
    Using a shift-share approach to instrument for state funding, I find that universities employ 4.4\% more lecturers per student following a 10\% funding cut.
    This shift is accompanied by a reduction in assistant and full professors by 1.4\% and 1.2\%  per student, respectively.
    These effects are concentrated to 1990--2009, with waning national trends for 2010 and onwards.
    Incumbent professors' salaries, promotion rates, and quit rates at Illinois universities remain unaffected by large funding cuts in the 2010s, indicating that the substitution arose from limiting the hiring of new professors.
    Stagnating state funding impacts public universities and faculty, likely contributing to deteriorating student outcomes at public universities since the 1990s. \\ \\

    & \textbf{Food Insecurity Among Military Veterans (2025).}

    Joint with Seungmin Lee (Notre Dame), Chris Barrett, John Hoddinott (Cornell), Matthew Rabbitt (USDA).

    (Draft Forthcoming) \\ \\

    %\textbf{Work in Progress} &
    %1. The labour market effects of genetics and mental health
    %(preliminary stages).
%
    %2. Selection-into-Treatment?
    %Judging quasi-experimental research designs in the real world
    %(preliminary stages). \\ \\

    \vfill \textbf{Other}
    & \vfill \textbf{Personal Interests}: Yoga, road cycling, open source software.
    
    \textbf{Programming}: R, Python, Bash, \LaTeX.
\end{longtable}
\end{document}

% Below is old code, which can be returned to after the job market.

%% Next page: superfluous list of things, including papers
%\newpage
%\onehalfspacing
%\restoregeometry
%\newgeometry{left=0.75in, right=0.75in, top=1in}

%%%%%%%%%%%%%%%%%%%%%%%%%%%%%%%%%%%%%%
%% Current Projects
%\section*{Work in Progress}
%\begin{enumerate}[itemsep=2.5pt, label={}]
%    \item \bibentry{direct-indirect-education}
%    \item \bibentry{veterans-food-insec}
%    %\item \bibentry{domestic-food-sec}
%\end{enumerate}

%%%%%%%%%%%%%%%%%%%%%%%%%%%%%%%%%%%%%%
%% Publications
%\section*{Publications}
%\begin{enumerate}[itemsep=2.5pt, label={}]
%    \item NONE YET.
%\end{enumerate}

%%%%%%%%%%%%%%%%%%%%%%%%%%%%%%%%%%%%%%
%% Working papers.
% \section*{Working Papers}
% \begin{enumerate}[itemsep=2.5pt, label={}]
%     \item \bibentry{mediation-natural-experiment}
%     \item \bibentry{state-funding-faculty}
%     \item \bibentry{osrs2022}
% \end{enumerate}
% \bibliographystyle{apalike-edit}
% \nobibliography{bibliography.bib}
